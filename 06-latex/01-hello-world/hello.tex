% How to prepare Doomemacs for LaTeX (LuaTeX)
% sudo dnf install texlive-full
% sudo dnf install texlive-babel-russian
% also I did (on Windows):
% tlmgr  install fontspec enumitem extsizes subfig \
%                caption natbib babel-russian titlesec lh cyrillic \
%                hyphen-russian
% delete ~/.texlive2024 folder
% in Doomemacs enable pdf (build pdf-tool) M-x pdf-tools-install
%                                          M-x pdf-info-check-epdfinfo
\documentclass[oneside,final,14pt]{extreport}
\usepackage[english,russian]{babel}
\usepackage{fontspec} % Advanced font management (works only with LuaTeX)
\setmainfont{JetBrains Mono}
\newfontfamily\jbmono{JetBrains Mono} % Replace 'JetBrains Mono' with your desired font
\begin{document}
\part{Начало}
\chapter{Настраивам LaTeX}
\section{Как выводить текст, делать переносы, абзацы}
\subsection{Вывод текста простой}
\subsubsection{Ну т.е. такой простой вывод, когда мы все по умолчанию оставляем}
\paragraph{Как есть так и оставляем}
\subparagraph{Ничего не добавляя}
Если хочешь что бы не было отсупа, то используй комаду
\textbf{noindent} в начале строки.\newline
\subsection{Вывод специальных символов}
Следующие символы можно экранировать обратным слешем
перед ними: \% \{ \} \$ \& \# \_ \\
Дополнительное экранирование для особых
символов: \textbackslash  \textasciicircum \~{}  `` '' \\

\subsection{Размеры шрифта, относительные}
\tiny oчень маленький шрифт, трудно читаемый \\
\scriptsize малеький шрифт \\
\footnotesize таким размером обычно делают сноски \\
\small шрифт почти такой же как и обычный \\
\normalsize нормальный размер букв \\
\large чуть увеличенный размер \\
\Large еще более увеличенный размер \\
\LARGE очень большой размер текста \\
\huge просто огромный размер текста \\
\Huge нереально огромный размер текста \\
\Huge to large some string of some text \\
\normalsize

\subsection{Вывод формул, примеры}
% Example 1
\ldots when Einstein introduced his formula
\begin{equation}
    e = m \cdot c^2 \; ,
\end{equation}
which is at the same time the most widely known
and the least well understood physical formula.
% Example 2
\ldots from which follows Kirchhoff's current law:
\begin{equation}
    \sum_{k=1}^{n} I_k = 0 \; .
\end{equation}
Kirchhoff's voltage law can be derived \ldots
% Example 3
\ldots which has several advantages.
\begin{equation}
    I_D = I_F - I_R
\end{equation}
is the core of a very different transistor model. \ldots
\tableofcontents
\end{document}


% Local Variables:
% TeX-engine: luatex
% End:
